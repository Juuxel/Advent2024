\documentclass[a4paper]{article}

\usepackage{amsmath}
\usepackage{amssymb}

\renewcommand{\geq}{\geqslant}

\title{Advent of Code 2024 -- day 11 notes}
\date{11 December 2024}
\author{}

\begin{document}
\maketitle

We can optimise the algorithm by replacing a ``raw'' number \(n\) with the representation
\(n = r \cdot 2024^k\), where \(r\) is a nonnegative long, \(\gcd(r, 2024) = 1\) and \(k\) is a nonnegative int.

The number of digits of \(n\), denoted as \(D(n)\), can also be computed using
just \(r\) and \(k\) without having to compute
the actual value of \(n\):
\[
    D(r \cdot 2024^k) = \lfloor k \log_{10} 2024 + \log_{10} r \rfloor + 1.
\]

The rules then turn to this:
\begin{enumerate}
    \item If \(r = 0\), yield \((r', k') = (1, 0)\).
    \item If \(D(n)\) is even, we have to compute the product to split \(n\)
    into the two halves. We'll denote them as \(L\) and \(R\).
    \begin{enumerate}
        \item \(L = \Bigl\lfloor \frac{n}{10^{D(n)/2}} \Bigr\rfloor\)
        \item \(R = n - L \cdot 10^{D(n)/2}\)
        \item Yield the \((r, k)\)-representations of \(L\) and \(R\).
    \end{enumerate}
    \item Otherwise, yield \((r', k') = (r, k + 1)\).
\end{enumerate}

In particular, we want to simplify step 2. We can replace
\(H = r\cdot 2024^k\), one of the halves, with a smaller number \(H' = r' \cdot 2024^{k'}\) that needs to
satisfy the following:
\begin{enumerate}
    \item \(r' = 0\) if and only if \(r = 0\)
    \item The number of digits of \(H' \cdot 2024^\ell\) is even
    if and only if \(D(H \cdot 2024^\ell)\) is even for all \(\ell \geq 0\).
    Written as a congruence,
    \[\begin{split}
        D(H \cdot 2024^\ell) &\equiv D(H' \cdot 2024^\ell)\mod{2} \\
        \Leftrightarrow D(r \cdot 2024^{k + \ell}) &\equiv D(r' \cdot 2024^{k' + \ell})\mod{2} \\
        \Leftrightarrow \lfloor (k + \ell) \log_{10} 2024 + \log_{10} r \rfloor &\equiv \lfloor (k' + \ell) \log_{10} 2024 + \log_{10} r' \rfloor\mod{2}. \\
    \end{split}\]
\end{enumerate}

\end{document}
